%!TEX root = ../graduate-work.tex

\section{Классы эквивалентных ключей в случае \texorpdfstring{$u=2$}{u=2}.}

Для случая $u=2$ задача поиска эквивалентных ключей криптосистемы
Мак-Элиса--Сидельникова основывается на изучении множеств
$\mathcal G(H_1,H_2)$, которые определяются так:

\begin{eqnarray*}
\mathcal G(H_1,H_2)&=\{\Gamma\in S_{2n}|\exists A_1,A_2\text{ ---
невырожденные $k\times k$-матрицы такие, что}&\\
&(H_1R\|H_2R)\Gamma=(A_1 R\|A_2 R)\}&
\end{eqnarray*}

Как указывалось в предыдущем параграфе вопрос описания множеств
$\mathcal G(H_1,\ldots,H_u)$ тесно связан с множествами $\mathcal
L(H_1,\ldots,H_u)$. Сформулируем их определение для случая $u=2$.
\begin{eqnarray*}
\mathcal L(H_1,H_2)&=&\{(A_1,A_2)\text{ --- кортеж
невырожденных матриц,}\\
&&\text{для которых существует перестановка }\Gamma\in S_{2n}:\\
&&(H_1R\|H_2R)\Gamma=(A_1R\|A_2R).\}
\end{eqnarray*}

Если матрица $H^{-1}_1H_2$ (или $H^{-1}_2H_1$) принадлежит
множеству $\mathcal A(RM(r,m))$, то описание данного множества
даёт утверждение~\ref{prop2}. Интересен случай, когда эта матрица
не принадлежит множеству $\mathcal A(RM(r,m))$.

Перепишем соотношение, задающее множества $\mathcal G(H_1,H_2)$ и
$\mathcal L(H_1,H_2)$, в следующем виде
$$
(R\|H^{-1}_1H_2R)\Gamma=(H^{-1}_1A_1R\|H^{-1}_1A_2R).
$$
Введём новые матрицы: $T=H^{-1}_1H_2$, $X=H^{-1}_1A_1$,
$Y=H^{-1}_1A_2$. В новых обозначения последнее соотношение
перепишется так
$$
(R\|TR)\Gamma=(XR\|YR)\eqno(\star).
$$
Если в этом соотношении фиксировать матрицу $T$, то можно получить
матричное уравнение относительно $X,Y,\Gamma$. Следующее
утверждение устанавливает связь решений уравнения ($\star$) с
множествами $\mathcal G(H_1,H_2)$ и $\mathcal L(H_1,H_2)$.
\begin{proposition}\label{prop13}
Пусть тройка $(X,Y,\Gamma)$, $X,Y\in GL_k(F_2)$, $\Gamma$ ---
перестановка на множестве из $2n$ элементов, является решением
уравнения
$$
(R\|TR)\Gamma=(XR\|YR).
$$
Тогда для любой невырожденной матрицы $H$ размера $k\times k$ пара
$(HX,HY)\in\mathcal L(H,HT)$ и $\Gamma\in\mathcal G(H,HT)$.

Справедливо и обратное. Если пара $(A_1,A_2)$ принадлежит
множеству $\mathcal L(H_1,H_2)$, а $\Gamma\in \mathcal
G(H_1,H_2)$, причём
$$
(H_1R\|H_2R)\Gamma=(A_1R\|A_2R),
$$
то тройка $(H^{-1}_1A_1,H^{-1}_1A_2,\Gamma)$ является решением
уравнения
$$
(R\|TR)\Gamma=(XR\|YR),\text{ при } T=H^{-1}_1H_2.
$$
\end{proposition}
\begin{proof}
Справедливость данного утверждения следует из определения
соответствующих множеств.
\end{proof}

Утверждение~\ref{prop13} позволяет сводить задачу описания множеств
$\mathcal L(H_1,H_2)$ к решению уравнения
$$
(R\|TR)\Gamma=(XR\|YR).
$$

Рассмотрим матрицы $T_{\widetilde\alpha}$ следующего типа:
$$
T_{\widetilde\alpha}=
\begin{pmatrix}
1&\alpha_1&\ldots&\alpha_{k-1}\\
0&1&\ldots&0\\
\vdots&\vdots&\ldots&\vdots\\
0&0&\ldots&1\\
\end{pmatrix},
$$
здесь $\widetilde\alpha=(\alpha_1,\ldots,\alpha_{k-1})$. Очевидно,
что для любого набора $\widetilde\alpha$ матрица
$T_{\widetilde\alpha}$ невырождена, поэтому можно рассмотреть
уравнение $(\star)$ с матрицей $T$ равной $T_{\widetilde\alpha}$.
Отметим также, что матрица $T_{\widetilde\alpha}$ при
$\widetilde\alpha\ne 0$ не принадлежит множеству $\mathcal
A(RM(r,m))$, поскольку при $\widetilde\alpha\ne 0$ в матрицах $R$
и $T_{\widetilde\alpha}R$ веса их первых строчек будут отличаться.

Занумеруем всё множество столбцов порождающей матрицы $R$ кода
Рида--Маллера $RM(r,m)$ числами от $1$ до $n$. Обозначим символом
$\mathcal N$ множество $\{1,2,\ldots,n\}$. \Def{\emph{Носителем}
$n$-мерного вектора $x$ над полем $F_2$ называется множество всех
тех номеров координат, в которых у вектора $x$  стоит $1$, то есть
$$
\supp(x)=\{i\in\mathcal N|x_i=1\}.
$$ }
Множество дополнительное к носителю вектора $x$ будем обозначать
символом $\overline{\supp}(x)$, то есть
$$
\overline{\supp}(x)=\mathcal N\setminus \supp(x).
$$
Заметим, что множество $\overline{\supp}(x)$ состоит из тех номеров
$i=1,\ldots,n$, для которых $x_i=0$. \Def{Пусть $I\subset\mathcal
N$, тогда \emph{укороченным кодом} Рида--Маллера $RM_I(r,m)$ будем
называть множество кодовых слов $RM(r,m)$, у которых на местах с
номерами из множества $I$ стоят нули, то есть
$$
RM_I(r,m)=\{x\in RM(r,m)|\forall i\in I\;x_i=0\}.
$$}
Легко проверить, что данное множество действительно является
кодом, то есть линейным подпространством пространства $F_2^n$.

\Def{Обозначим через $R^i$ для некоторого $i$ матрицу,
получающуюся из матрицы $R$ выкидыванием $i$-той строки. Кодом
Рида--Маллера $RM^i(r,m)$ \emph{с выбрасыванием} назовём код,
порождающей матрицей которого является $R^i$.}

Везде далее $RM^{i}_I(r,m)=(RM^i)_I(r,m)$, то есть $RM^i_I(r,m)$
--- код, получающийся из кода Рида--Маллера $RM(r,m)$ сначала
выбрасыванием $i$-той строки из порождающей матрицы, а потом
укорачиванием нового кода в множестве координат $I$.

\begin{proposition}\label{prop20}
Перестановка $\Gamma$ принадлежит множеству $\mathcal
G(E,T_{\widetilde\alpha})$, если и только если её можно представить
в виде $\widetilde\Gamma(\Gamma_1\|\Gamma_2)$, где
$\Gamma_1,\Gamma_2\in Aut(RM(r,m))$, а перестановка
$\widetilde\Gamma$
\begin{itemize}
\item[1)]выделяет блоки одинаковых столбцов и переводит их друг в
друга;
\item[2)] выбирает некоторый вектор $x$ из кода
$RM^1_{\supp((1,\widetilde\alpha)R)}(r,m)$ и переводит столбцы
$R_i$ в $T_{\widetilde\alpha}R_i$ для любого $i\in \supp(x)$.
\end{itemize}
\end{proposition}
\begin{proof}
Рассмотрим уравнение $(\star)$ :
$$
(R\|T_{\widetilde\alpha}R)\Gamma=(XR\|YR).
$$
В силу того, что $R$ является стандартной формой порождающей матрицы
кода $RM(r,m)$, то первая её строка состоит из всех единиц,
поэтому любой столбец можно представить в виде $\binom{1}{R'_i}$,
$i=1,2,\ldots,n$.

Для начала заметим, что матрица $T_{\widetilde\alpha}R$ состоит из
столбцов двух видов. Первый тип --- такие же столбцы как и в
матрице $R$, то есть они имеют вид $\binom{1}{R'_i}$. Таких
столбцов будет столько, сколько единиц в первой строке матрицы
$T_{\widetilde\alpha}R$, то есть $|\supp((1,\widetilde\alpha)R)|$.
Второй тип
--- столбцы, которые отличаются от какого-нибудь столбца в матрице
$R$ только в первой координате. Их число, очевидно, равно числу
нулей в первой строке, то есть
$|\overline{\supp}((1,\widetilde\alpha)R)|$. Такие столбцы имеют
вид $\binom{0}{R'_i}$.

Возьмём какое нибудь решение $(X,Y,\Gamma)$ уравнения $(\star)$.
После действия перестановки $\Gamma$ никакие столбцы из
$T_{\widetilde\alpha}R$ первого типа не могут оказаться в одной
матрице ($XR$ или $YR$) с точно такими же столбцами из матрицы
$R$, в силу того, что ни в одной порождающей матрице кода
Рида--Маллера нет двух одинаковых столбцов.

Докажем теперь, что в матрицах $XR$ и $YR$ не могут лежать сразу
столбцы следующих видов:$\binom{0}{R'_i}$, $\binom{1}{R'_i}$.
Действительно, пусть это имеет место. В матрицах $R$ и
$T_{\widetilde\alpha}R$ первый столбец одинаковый и равен
$\binom{1}{0}$, поэтому и в матрице $XR$ и в матрице $YR$
обязательно должен быть столбец $\binom{1}{0}$. Но тогда, сложив
$\binom{0}{R'_i}$, $\binom{1}{R'_i}$ и $\binom{1}{0}$, получим:
$$
\binom{0}{R'_i}\oplus\binom{1}{R'_i}\oplus\binom{1}{0}=0.
$$
Но последнее означает, что данная система столбцов линейно
зависимая. Откуда немедленно следует, что кодовое расстояние
дуального к коду с порождающей матрицей $XR$ или $YR$ меньше, либо
равно $3$. Но $XR$ и $YR$ --- суть коды Рида--Маллера $RM(r,m)$,
$r\geqslant 2$. Известно~\cite{McWilliams}, что кодовое расстояние
кода $RM^{\perp}(r,m)$ в точности равно $2^{r+1}$. Учитывая это
получаем, что $2^{r+1}\leqslant 3$. Что невозможно при $r\geqslant
2$. Получили противоречие.

Из всего выше сказанного следует, что перестановка $\Gamma$ лишь
может
\begin{itemize}
\item[1)] менять порядок следования столбцов внутри матриц $R$ и
$T_{\widetilde\alpha}R$; \item[2)] выделять группы столбцов в
одной матрице и перекидывать их в те же самые столбцы в другой
матрице; \item[3)] выделять группы столбцов второго типа
$\binom{0}{R'_{i_1}}$,
$\binom{0}{R'_{i_2}}$,\ldots,$\binom{0}{R'_{i_s}}$ в матрице
$T_{\widetilde\alpha}R$ и менять её с группой столбцов
$\binom{1}{R'_{i_1}}$,$\binom{1}{R'_{i_2}}$,\ldots
,$\binom{1}{R'_{i_s}}$ матрицы $R$.
\end{itemize}
Итак, предполагая что $(X,Y,\Gamma)$ --- решение
уравнения~($\star$), мы получили необходимые условия на
перестановку $\Gamma$. В этих же предположениях уточним условие
$3)$. Возьмём перестановку $\Gamma$, которая обязательно реализует
$3)$. Пусть после применения $\Gamma$ из матрицы
$(R\|T_{\widetilde\alpha}R)$ образовалась матрица $(XR\|YR)$. $XR$
--- задаёт код Рида--Маллера $RM(r,m)$, значит существует вектор
$\beta$ такой, что $\beta XR=\mathbf 1=(1,1,\ldots,1)$. Докажем,
что первая координата вектора $\beta$ обязательно равна $1$.
Действительно, в силу вида перестановки $\Gamma$, матрицу $XR$
можно представить в виде
$$
XR=
\begin{pmatrix}
1&1&\ldots&0&\ldots&0&\ldots&1\\
R'_{j_1}&R'_{j_2}&\ldots&R'_{i_1}&\ldots&R'_{i_s}&\ldots&R'_{j_n}\\
\end{pmatrix}.
$$
Здесь столбцы $R_{i_1},\ldots,R_{i_s}$ стоят на местах с номерами
$j_{i_1},\ldots, j_{i_s}$ соответственно. Причём матрица
$(R'_{j_1}R'_{j_2}\ldots R'_{i_1}\ldots R'_{i_s}\ldots R'_{j_n})$
получается  из матрицы \\ $(R'_1 R'_2 \ldots R'_{i_1} \ldots
R'_{i_s} \ldots R'_n)$ применением некоторой перестановки
$\Gamma_1\in S_n$. Но никакая сумма строк матрицы $(R'_1\ldots
R'_n)$ не может равняться единичной строчке $\mathbf 1$.
Следовательно, сумма строк матрицы $XR$, дающая единичную строчку,
должна обязательно содержать строку с первым номером, то есть
$\beta_1=1$. Учитывая, что $\beta_1=1$, получаем
\begin{multline*}
\mathbf 1=\beta XR=\\=(1\;1\;\ldots\;0\;\ldots\;0\;\ldots\;1)\oplus
(\beta_2,\ldots,\beta_k)(R'_{j_1}\;R'_{j_2}\;\ldots\;R'_{i_1}\;\ldots\;R'_{i_s}\;\ldots\;R'_{j_n}).    
\end{multline*}
Откуда
$$
(\beta_2,\ldots,\beta_k)(R'_{j_1}\;R'_{j_2}\;\ldots\;R'_{i_1}\;\ldots\;R'_{i_s}\;\ldots\;R'_{j_n})=(0\;0\;\ldots\;1\;\ldots\;1\;\ldots\;0).
$$
Здесь в векторе $(0\;0\;\ldots\;1\;\ldots\;1\;\ldots\;0)$ единицы
стоят на позициях с номерами $j_{i_1},j_{i_2},\ldots,j_{i_s}$.
Умножим левую и правую часть этого равенства на перестановку
$\Gamma^{-1}_1$, тогда
$$
(\beta_2,\ldots,\beta_k)(R'_1\;\ldots\;R'_n)=(0\;0\;\ldots\;1\;\ldots\;1\;\ldots\;0).\eqno(\divideontimes)
$$
Здесь уже в векторе $(0\;0\;\ldots\;1\;\ldots\;1\;\ldots\;0)$
единицы стоят на позициях с номерами $i_1,\ldots,i_s$. Обозначим
через $x$ вектор, стоящий в правой части последнего равенства, то
есть
$$
x=(0\;0\;\ldots\;1\;\ldots\;1\;\ldots\;0),\;x_j=1\Leftrightarrow
j=i_p (p=1,\ldots,s).
$$
Введя такое обозначение, соотношение~($\divideontimes$) можно
записать так:
$$
(0,\beta_2,\ldots,\beta_k)R=x.
$$
Откуда немедленно следует, что вектор $x$, принадлежит коду
$RM^1(r,m)$. Заметим, что $i_1,\ldots,i_s\in
\overline{\supp}((1,\widetilde\alpha)R)$, то есть
$i_1,\ldots,i_s\not\in \supp((1,\widetilde\alpha)R)$. Этот факт
означает, что вектор $x\in
RM^1_{\supp((1,\widetilde\alpha)R)}(r,m)$.

Введём матрицу $K$:
$$
K=\begin{pmatrix}
1&\beta_2&\ldots&\beta_k\\
0&1&\ldots&0\\
\vdots&\vdots&\ldots&\vdots\\
0&0&\ldots&1\\
\end{pmatrix}.
$$
В результате матрицы $XR$ и $YR$ можно представить в виде
$$
XR=K\cdot R\Gamma_1;\;YR=K\cdot
T_{\widetilde\alpha}R\Gamma_2.\eqno(3)
$$
Последнее верно, так как строчки с номерами $2,\ldots,k$ матрицы
$T_{\widetilde\alpha}R$ не отличаются от соответствующих строчек
матрицы $R$. В силу этого, строчки $2,\ldots,k$ матрицы $K\cdot
T_{\widetilde\alpha}R$ также будут совпадать с соответствующими
строками матрицы $R$. Первая же строка в $K\cdot
T_{\widetilde\alpha}R$ будет получаться прибавлением вектора $x$ к
первой строке матрицы $T_{\widetilde\alpha}R$. В векторе $x$
единицы стоят только на тех местах, на которых в первой строке
матрицы $T_{\widetilde\alpha}R$ стоят нули, поэтому замена
столбцов вида $\binom{0}{R_{i_p}}$ на столбцы
$\binom{1}{R_{i_p}}$($p=1,\ldots,s$) в матрице
$T_{\widetilde\alpha}R$ эквивалентна прибавлению вектора
$x$($x_{j}=1 \Leftrightarrow j=i_p(p=1,\ldots,s)$) к первой строке
матрицы $T_{\widetilde\alpha}R$. Поэтому матрица $K\cdot
T_{\widetilde\alpha}R$ будет состоять из тех же столбцов, что и
матрица $YR$. И для некоторой перестановки $\Gamma_2$ будет
выполняться
$$
YR=K\cdot T_{\widetilde\alpha}R\Gamma_2.
$$

Из соотношения~(3) следует, что каждая из перестановок
$\Gamma_1,\Gamma_2$ является автоморфизмов кода $RM(r,m)$, а
матрицы $X$ и $Y$ имеют вид
$$
X=KD_1,\;Y=KD_2,\;D_1,D_2\in\mathcal A(RM(r,m)).
$$

Итак, из всей цепочки рассуждений видно, что матрицу $\Gamma$
можно разложить в произведение некоторой матрицы
$\widetilde\Gamma$ на блоковую перестановку
$(\Gamma_1\|\Gamma_2)$, состоящую из автоморфизмов кода $RM(r,m)$.
Тем самым можно не только уточнить условия 1),2),3), но и
сформулировать их уже для перестановки $\widetilde\Gamma$. В
итоге, $\widetilde\Gamma$ может только
\begin{itemize}
\item[1')] выделять группы столбцов в одной матрице и перекидывать их
в те же самые столбцы в другой матрице;
\item[2')] выделять группы столбцов второго типа $\binom{0}{R'_{i_1}}$,
$\binom{0}{R'_{i_2}}$,\ldots,$\binom{0}{R'_{i_s}}$ в матрице $TR$
и менять её с группой столбцов
$\binom{1}{R'_{i_1}}$,$\binom{1}{R'_{i_2}}$,\ldots
,$\binom{1}{R'_{i_s}}$ матрицы $R$, при условии, что существует
вектор $x$ из кода $RM^1_{\supp((1,\widetilde\alpha)R)}(r,m)$
такой, что для любого $j=1,\ldots,s$ $x_{i_j}=1$.
\item[3')] выполнять 1') и 2') одновременно.
\end{itemize}

Докажем обратное. Пусть перестановку $\Gamma$ можно представить в
виде произведения перестановок $\widetilde\Gamma(\Gamma_1\|\Gamma_2)$, где
$\Gamma_1,\Gamma_2$ --- автоморфизмы кода Рида--Маллера $RM(r,m)$,
а $\widetilde\Gamma$ удовлетворяет условиям 1'),2'),3'). Если
$\widetilde\Gamma$ удовлетворяет только условию 1'), то, очевидно,
что $(D_1,D_2,\Gamma)$ (здесь $D_1R=R\Gamma_1$ и $D_2R=R\Gamma_2$)
--- решение уравнения $(\star)$, то есть
$\Gamma\in\mathcal G(E,T_{\widetilde\alpha})$. Пусть она реализует
ещё и 2'). Тогда существует кодовое слово $x\in
RM^1_{\supp((1,\widetilde\alpha)R)}(r,m)$ такое, что перестановка
$\widetilde\Gamma$ переводит друг в друга столбцы $R_i$ и
$T_{\widetilde\alpha}R_i$ для любого номера $i\in \supp(x)$. Так
как $x\in RM^1_{\supp((1,\widetilde\alpha)R)}(r,m)$, то, по
определению кода $RM^1_{\supp((1,\widetilde\alpha)R)}(r,m)$,
существует $(k-1)$-мерный вектор $\beta=(\beta_2,\ldots,\beta_k)$
такой, что $(0,\beta)R=x$ и $(0,\beta)T_{\widetilde\alpha}R=x$.
Построим матрицу $X$ следующим образом
$$
X=\begin{pmatrix}
1&\beta_2&\ldots&\beta_k\\
0&1&\ldots&0\\
\vdots&\vdots&\ldots&\vdots\\
0&0&\ldots&1\\
\end{pmatrix},\eqno(\ast)
$$
и пусть $Y=XT_{\widetilde\alpha}$. Тогда, очевидно,
$(X,Y,\widetilde\Gamma)$
--- решение уравнения $(\star)$, а значит и тройка $(XD_1,YD_2,\Gamma)$
(здесь $D_1R=R\Gamma_1$ и $D_2R=R\Gamma_2$) тоже решение уравнения
$(\star)$, поэтому $\Gamma\in\mathcal G(E,T_{\widetilde\alpha})$.

Тем самым утверждение полностью доказано.
\end{proof}

Из утверждения~\ref{prop20} как следствие можно получить
утверждение~\ref{prop21}.

\begin{proposition}\label{prop21}
Пусть код Рида--Маллера $RM(r,m)$ имеет параметры: $r\geqslant 2$,
$r<m$. Обозначим через $R$ стандартную форму его порождающей
матрицы. Тогда число $A$-классов в множестве $\mathcal
L(E,T_{\widetilde\alpha})(\widetilde\alpha\neq\widetilde 0)$ равно
$$
2^{\dim \left[RM^1_{\supp((1,\widetilde\alpha)R)}(r,m)\right]-1}.
$$
\end{proposition}
\begin{proof}
В силу утверждения~\ref{prop20} любую перестановку
$\Gamma\in\mathcal G(E,T_{\widetilde\alpha})$ можно представить в
виде $\widetilde\Gamma(\Gamma_1\|\Gamma_2)$, где
$\Gamma_1,\Gamma_2$ --- автоморфизмы кода Рида--Маллера $RM(r,m)$,
а $\widetilde\Gamma$ задаётся некоторым словом из
$RM^1_{\supp((1,\widetilde\alpha)R)}(r,m)$. Тем самым число
перестановок в множестве $\mathcal G(E,T_{\widetilde\alpha})$
равно произведению числа кодовых слов в
$RM^1_{\supp((1,\widetilde\alpha)R)}(r,m)$ на $|Aut(RM(r,m))|^2$ и
на
$$|\Gamma_{\phi}(E,T_{\widetilde\alpha})|=2^{|\supp((1,\widetilde\alpha)R)|}.$$
Из утверждения~\ref{prop1} следует, что мощность множества
$\mathcal L(E,T_{\widetilde\alpha})$ в
$\mathcal L(E,T_{\widetilde\alpha})\geqslant 2^{|\supp((1,\widetilde\alpha)R)|}$ раз меньше, чем $|\mathcal
G(E,T_{\widetilde\alpha})|$. Значит
$$
|\mathcal L(E,T_{\widetilde\alpha})|=2^{\dim
\left[RM^1_{\supp((1,\widetilde\alpha)R)}(r,m)\right]}|Aut(RM(r,m))|^2.
$$
Осталось заметить, что, так как $T_{\widetilde\alpha}$ не
принадлежит множеству $\mathcal A(RM(r,m))$, мощность каждого
$A$-класса равна $2|Aut(RM(r,m))|^2$. Итак, число $A$-классов в
$\mathcal L(E,T_{\widetilde\alpha})$ равно
$$
\frac{2^{\dim
\left[RM^1_{\supp((1,\widetilde\alpha)R)}(r,m)\right]}|Aut(RM(r,m))|^2}{2|Aut(RM(r,m))|^2}=2^{\dim
\left[RM^1_{\supp((1,\widetilde\alpha)R)}(r,m)\right]-1}.
$$
Что и требовалось доказать.
\end{proof}

\noindent\textbf{Пример.} Возьмём код Рида--Маллера с параметрами
$r=2$, $m=3$. В качестве матрицы $T$ выберем следующую:
$$
T=\begin{pmatrix}
1&1&1&0&1&0&0\\
0&1&0&0&0&0&0\\
0&0&1&0&0&0&0\\
0&0&0&1&0&0&0\\
0&0&0&0&1&0&0\\
0&0&0&0&0&1&0\\
0&0&0&0&0&0&1\\
\end{pmatrix}
$$
Тогда уравнение $(\star)$ примет вид: \vspace{10pt}

\begin{tabular}{||c|c||}
\begin{tabular}{cccccccc}
1&1&1&1&1&1&1&1\\
0&0&0&0&1&1&1&1\\
0&0&1&1&0&0&1&1\\
0&1&0&1&0&1&0&1\\
0&0&0&0&0&0&1&1\\
0&0&0&0&0&1&0&1\\
0&0&0&1&0&0&0&1\\
\end{tabular}&
\begin{tabular}{cccccccc}
1&1&0&0&0&0&0&0\\
0&0&0&0&1&1&1&1\\
0&0&1&1&0&0&1&1\\
0&1&0&1&0&1&0&1\\
0&0&0&0&0&0&1&1\\
0&0&0&0&0&1&0&1\\
0&0&0&1&0&0&0&1\\
\end{tabular}
\end{tabular}$\;\Gamma=\|XR|YR\|$.
\vspace{10pt}

Вектор $\widetilde\alpha$ будет равен $(1,1,0,1,0,0)$. Множество
$$\supp((1,1,1,0,1,0,0)R=(1,1,0,0,0,0,0,0))$$ состоит из двух
координат $1$ и $2$. Код $RM^1_{1,2}$ будет иметь порождающую
матрицу $R_{1,2}$:
$$
R_{1,2}=\begin{pmatrix}
0&0&0&0&1&1&1&1\\
0&0&1&1&0&0&1&1\\
0&0&0&0&0&0&1&1\\
0&0&0&0&0&1&0&1\\
0&0&0&1&0&0&0&1\\
\end{pmatrix}
$$

Построим, например, перестановки $\widetilde\Gamma_1$ и
$\widetilde\Gamma_2$,$\widetilde\Gamma_3$ которые будут
соответствовать первому, второму и последнему вектору матрицы
$R_{1,2}$.
$$
\widetilde\Gamma_1=(1)(2)(3)(4)(9)(10)(11)(12)(5\;13)(6\;14)(7\;15)(8\;16).
$$
$$
\widetilde\Gamma_2=(1)(2)(3)(4)(5)(6)(9)(10)(11)(12)(13)(14)(7\;15)(8\;16).
$$
$$
\widetilde\Gamma_3=(1)(2)(3)(5)(6)(8)(9)(10)(11)(13)(14)(4\;12)(8\;16).
$$
Соответствующие матрицы $(X_1,Y_1)$, $(X_2,Y_2)$, $(X_3,Y_3)$ будут
выглядеть так
$$
X_1=\begin{pmatrix}
1&1&0&0&0&0&0\\
0&1&0&0&0&0&0\\
0&0&1&0&0&0&0\\
0&0&0&1&0&0&0\\
0&0&0&0&1&0&0\\
0&0&0&0&0&1&0\\
0&0&0&0&0&0&1\\
\end{pmatrix}
Y_1=X_1T=\begin{pmatrix}
1&0&1&0&1&0&0\\
0&1&0&0&0&0&0\\
0&0&1&0&0&0&0\\
0&0&0&1&0&0&0\\
0&0&0&0&1&0&0\\
0&0&0&0&0&1&0\\
0&0&0&0&0&0&1\\
\end{pmatrix}
$$
$$
X_2=\begin{pmatrix}
1&0&1&0&0&0&0\\
0&1&0&0&0&0&0\\
0&0&1&0&0&0&0\\
0&0&0&1&0&0&0\\
0&0&0&0&1&0&0\\
0&0&0&0&0&1&0\\
0&0&0&0&0&0&1\\
\end{pmatrix}
Y_2=X_2T=\begin{pmatrix}
1&1&0&0&1&0&0\\
0&1&0&0&0&0&0\\
0&0&1&0&0&0&0\\
0&0&0&1&0&0&0\\
0&0&0&0&1&0&0\\
0&0&0&0&0&1&0\\
0&0&0&0&0&0&1\\
\end{pmatrix}
$$
$$
X_3=\begin{pmatrix}
1&0&0&0&0&0&1\\
0&1&0&0&0&0&0\\
0&0&1&0&0&0&0\\
0&0&0&1&0&0&0\\
0&0&0&0&1&0&0\\
0&0&0&0&0&1&0\\
0&0&0&0&0&0&1\\
\end{pmatrix}
Y_3=X_3T=\begin{pmatrix}
1&1&1&0&1&0&1\\
0&1&0&0&0&0&0\\
0&0&1&0&0&0&0\\
0&0&0&1&0&0&0\\
0&0&0&0&1&0&0\\
0&0&0&0&0&1&0\\
0&0&0&0&0&0&1\\
\end{pmatrix}
$$
При этом
$$
X_1R=
\begin{pmatrix}
1&1&1&1&0&0&0&0\\
0&0&0&0&1&1&1&1\\
0&0&1&1&0&0&1&1\\
0&1&0&1&0&1&0&1\\
0&0&0&0&0&0&1&1\\
0&0&0&0&0&1&0&1\\
0&0&0&1&0&0&0&1\\
\end{pmatrix}\;
Y_1R=
\begin{pmatrix}
1&1&0&0&1&1&1&1\\
0&0&0&0&1&1&1&1\\
0&0&1&1&0&0&1&1\\
0&1&0&1&0&1&0&1\\
0&0&0&0&0&0&1&1\\
0&0&0&0&0&1&0&1\\
0&0&0&1&0&0&0&1\\
\end{pmatrix}\;
$$
$$
X_2R=
\begin{pmatrix}
1&1&0&0&1&1&0&0\\
0&0&0&0&1&1&1&1\\
0&0&1&1&0&0&1&1\\
0&1&0&1&0&1&0&1\\
0&0&0&0&0&0&1&1\\
0&0&0&0&0&1&0&1\\
0&0&0&1&0&0&0&1\\
\end{pmatrix}\;
Y_2R=
\begin{pmatrix}
1&1&1&1&0&0&1&1\\
0&0&0&0&1&1&1&1\\
0&0&1&1&0&0&1&1\\
0&1&0&1&0&1&0&1\\
0&0&0&0&0&0&1&1\\
0&0&0&0&0&1&0&1\\
0&0&0&1&0&0&0&1\\
\end{pmatrix}\;
$$
$$
X_3R=
\begin{pmatrix}
1&1&1&0&1&1&1&0\\
0&0&0&0&1&1&1&1\\
0&0&1&1&0&0&1&1\\
0&1&0&1&0&1&0&1\\
0&0&0&0&0&0&1&1\\
0&0&0&0&0&1&0&1\\
0&0&0&1&0&0&0&1\\
\end{pmatrix}\;
Y_3R=
\begin{pmatrix}
1&1&0&1&0&0&0&1\\
0&0&0&0&1&1&1&1\\
0&0&1&1&0&0&1&1\\
0&1&0&1&0&1&0&1\\
0&0&0&0&0&0&1&1\\
0&0&0&0&0&1&0&1\\
0&0&0&1&0&0&0&1\\
\end{pmatrix}\;
$$
Нетрудно видеть, что все эти матрицы задают различные $A$-классы.

Следствием предыдущих утверждений являются
утверждения~\ref{prop22} и~\ref{prop23}.

\begin{proposition}\label{prop22}
Для любой невырожденной матрицы $H$ перестановка $\Gamma$
принадлежит множеству $\mathcal G(H,HT_{\widetilde\alpha})$, если
и только если её можно представить в виде
$\widetilde\Gamma(\Gamma_1\|\Gamma_2)$, где $\Gamma_1,\Gamma_2\in
Aut(RM(r,m))$, а перестановка $\widetilde\Gamma$
\begin{itemize}
\item[1)]выделяет блоки одинаковых столбцов и переводит их  друг в
друга;
\item[2)] выбирает некоторый вектор $x$ из кода
$RM^1_{\supp((1,\widetilde\alpha)R)}(r,m)$ и переводит столбцы
$HR_i$ и $HT_{\widetilde\alpha}R_i$ друг в друга для любого $i\in
\supp(x)$.
\end{itemize}
\end{proposition}

\begin{proposition}\label{prop23}
Пусть код Рида--Маллера $RM(r,m)$ имеет параметры: $r\geqslant 2$,
$r<m$. Обозначим через $R$ стандартную форму его порождающей
матрицы. Тогда для любой невырожденной матрицы $H$ число
$A$-классов в множестве $\mathcal L(H,HT_{\widetilde\alpha})$
равно
$$
2^{\dim \left[RM^1_{\supp((1,\widetilde \alpha)R)}(r,m)\right]-1}.
$$
\end{proposition}

Рассмотрим теперь матрицу $T^{i}_{\tilde\alpha}(i>1)$ вида
$$
T^{i}_{\tilde\alpha}=\begin{pmatrix}
 &&         &          &       i           &                &\\
          &&         &          &       \downarrow           &&                \\
    &1     &        0&    \ldots&       0&     \ldots&               0&\\
    &0     &        1&    \ldots&       0&     \ldots&               0&\\
    &\vdots&   \vdots&    \ldots&  \vdots&     \ldots&          \vdots&\\
  i\rightarrow&\alpha_1& \alpha_2&    \ldots&       1&     \ldots&    \alpha_{k-1}&\\
    &\vdots&   \vdots&    \ldots&  \vdots&     \ldots&          \vdots&\\
     &    0&        0&    \ldots&       0&     \ldots&               1&\\
     &&&&&&
\end{pmatrix}.
$$

Для неё справедливо следующее утверждение.

\begin{proposition}\label{prop24}
Пусть
$I=\overline{\supp}(\overrightarrow{R}_i\oplus\overrightarrow{T^i_{\widetilde\alpha}R}_i)$,
здесь $\overrightarrow{R}_i$ и
$\overrightarrow{T^i_{\widetilde\alpha}R}_i$ --- $i$-тые строки
матриц $R$ и $T^i_{\widetilde\alpha}R$ соответственно. Тогда для
любого $x\in RM^i_I$ перестановка\\
$\Gamma=\Gamma'\Gamma''(\Gamma_1\|\Gamma_2)$, где
\begin{itemize}
\item[1)] $\Gamma'$ переводит одинаковые столбцы друг в друга;
\item[2)] $\Gamma''$ переводит друг в друга  столбцы $T^i_{\widetilde\alpha}R_j$
 и $R_j$ для любого $j\in \supp(x)$;
 \item[3)] $\Gamma_1,\Gamma_2$ --- автоморфизмы кода Рида--Маллера
 $RM(r,m)$;
\end{itemize}
принадлежит множеству $\mathcal G(E,T^i_{\widetilde\alpha})$.
\end{proposition}
\begin{proof}
Действительно, рассмотрим любое $x\in RM^i_I$. Из определения кода
$RM^i_I$ следует, что существует $k$-мерный вектор
$\beta=(\beta_1,\ldots,\beta_k)$ такой, что $\beta_i=0$ и $x=\beta
R$. Рассмотрим матрицу $K$:
$$
K=\begin{pmatrix}
 &&         &          &       i           &                &\\
          &&         &          &       \downarrow           &&                \\
    &1     &        0&    \ldots&       0&     \ldots&               0&\\
    &0     &        1&    \ldots&       0&     \ldots&               0&\\
    &\vdots&   \vdots&    \ldots&  \vdots&     \ldots&          \vdots&\\
  i\rightarrow&\beta_1& \beta_2&    \ldots&       1&     \ldots&    \beta_k&\\
    &\vdots&   \vdots&    \ldots&  \vdots&     \ldots&          \vdots&\\
     &    0&        0&    \ldots&       0&     \ldots&               1&\\
     &&&&&&
\end{pmatrix}.
$$
Теперь пусть перестановка $\Gamma''$ переводит друг в друга
столбцы $T^i_{\widetilde\alpha}R_j$ и $R_j$ для любого $j\in
\supp(x)$. Пусть после применения $\Gamma''$ матрица $R$ перешла в
матрицу $R'$, а матрица $T^i_{\widetilde\alpha}R$ --- в $R''$.
Отметим, что столбцы $T^i_{\widetilde\alpha}R_j$ и $R_j$ для
любого $j\in \supp(x)$ отличаются только в $i$-той координате. Тем
самым матрица $R'$ будет состоят из тех же строк, за исключением
$i$-той, что и $R$, а $i$-тая координата получается прибавлением
вектора $x$ к $i$-той строке матрицы $R$. Аналогичное будет
выполняться и для матрицы $R''$, то есть всё её строки, кроме
$i$-той, будут совпадать с соответствующими строками матрицы
$T^i_{\widetilde\alpha}R$, а в силу структуры
$T^i_{\widetilde\alpha}$ и со строками $R$. Строчка же с номером
$i$ получается в результате суммы вектора $x$ и $i$-той строки
матрицы $T^i_{\widetilde\alpha}R$. Поэтому действие перестановки
$\Gamma''$ можно промоделировать матрицей $K$ следующим образом:
$$
(R\|T^i_{\widetilde\alpha})\Gamma''=(R'\|R'')=K(R\|T^i_{\widetilde\alpha}R).
$$
А это и означает, что $\Gamma''\in\mathcal
G(E,T^i_{\widetilde\alpha})$.

Осталось заметить, что перестановки $\Gamma'$ и
$(\Gamma_1\|\Gamma_2)$ принадлежат множеству $\mathcal
G(E,T^i_{\widetilde\alpha})$ очевидным образом, и к тому же
$$
(R\|T^i_{\widetilde\alpha}R)\Gamma'=(R\|T^i_{\widetilde\alpha}R).
$$

Утверждение полностью доказано.
\end{proof}

\begin{proposition}\label{prop25}
Пусть код Рида--Маллера $RM(r,m)$ имеет параметры: $r\geqslant 2$,
$r<m$.  И пусть $T^i_{\widetilde\alpha}\not\in \mathcal
A(RM(r,m)).$ Обозначим через $R$ стандартную форму его порождающей
матрицы. Пусть также
$I=\overline{\supp}(\overrightarrow{R}_i\oplus\overrightarrow{T^i_{\widetilde\alpha}R}_i)$,
здесь $\overrightarrow{R}_i$ и
$\overrightarrow{T^i_{\widetilde\alpha}R}_i$ --- $i$-тые строки
матриц $R$ и $T^i_{\widetilde\alpha}R$ соответственно. Тогда число
$A$-классов в множестве $\mathcal L(E,T^i_{\tilde\alpha})$ не
меньше, чем
$$
2^{\dim \left[RM^i_I(r,m)\right]-1}.
$$
\end{proposition}
\begin{proof}
Из утверждения~\ref{prop24} следует, что
$$
|\mathcal G(E,T^i_{\widetilde\alpha})|\geqslant|\Gamma_{\phi}(E,T^i_{\widetilde\alpha})|2^{\dim
\left[RM^i_I(r,m)\right]}|Aut(RM(r,m))|^2.
$$
Откуда
$$
|\mathcal L(E,T^i_{\widetilde\alpha})|=\frac{|\mathcal
G(E,T^i_{\widetilde\alpha})|}{|\Gamma_{\phi}(E,T^i_{\widetilde\alpha})|}\geqslant
2^{\dim \left[RM^i_I(r,m)\right]}|Aut(RM(r,m))|^2.
$$
Так как по условию $T^i_{\widetilde\alpha}\not\in \mathcal
A(RM(r,m))$, то мощность каждого $A$-класса\\ равна
$2|Aut(RM(r,m))|^2$. Учитывая это, получаем требуемое неравенство.
\end{proof}

Для матриц $H$ и $HT^i_{\widetilde\alpha}$  можно сформулировать
утверждения, аналогичные утверждениям~\ref{prop24} и~\ref{prop25}.
Их справедливость является прямым следствием
утверждений~\ref{prop24} и~\ref{prop25}.

\begin{proposition}\label{prop26}
Пусть
$I=\overline{\supp}(\overrightarrow{R}_i\oplus\overrightarrow{T^i_{\widetilde\alpha}R}_i)$,
здесь $\overrightarrow{R}_i$ и
$\overrightarrow{T^i_{\widetilde\alpha}R}_i$ --- $i$-тые строки
матриц $R$ и $T^i_{\widetilde\alpha}R$ соответственно. Тогда для
любого $x\in RM^i_I$  и любой невырожденной матрицы $H$
перестановка $\Gamma=\Gamma'\Gamma''(\Gamma_1\|\Gamma_2)$, где
\begin{itemize}
\item[1)] $\Gamma'$ переводит одинаковые столбцы друг в друга;
\item[2)] $\Gamma''$ переводит друг в друга  столбцы $HT^i_{\widetilde\alpha}R_j$
 и $HR_j$ для любого $j\in \supp(x)$;
 \item[3)] $\Gamma_1,\Gamma_2$ --- автоморфизмы кода Рида--Маллера
 $RM(r,m)$;
\end{itemize}
принадлежит множеству $\mathcal G(H,HT^i_{\widetilde\alpha})$.
\end{proposition}

\begin{proposition}\label{prop27}
Пусть код Рида--Маллера $RM(r,m)$ имеет параметры: $r\geqslant 2$,
$r<m$.  И пусть $T^i_{\widetilde\alpha}\not\in \mathcal
A(RM(r,m)).$ Обозначим через $R$ стандартную форму его порождающей
матрицы. Пусть также
$I=\overline{\supp}(\overrightarrow{R}_i\oplus\overrightarrow{T^i_{\widetilde\alpha}R}_i)$,
здесь $\overrightarrow{R}_i$ и
$\overrightarrow{T^i_{\widetilde\alpha}R}_i$ --- $i$-тые строки
матриц $R$ и $T^i_{\widetilde\alpha}R$ соответственно. Тогда для
любой невырожденной матрицы $H$ число $A$-классов в множестве
$\mathcal L(H,HT^i_{\tilde\alpha})$ не меньше, чем
$$
2^{\dim \left[RM^i_I(r,m)\right]-1}.
$$
\end{proposition}

Из всех утверждений данного параграфа следует справедливость двух
основных теорем.
\begin{theorem}\label{theorem1}
Пусть $R$ --- стандартная форма порождающей матрицы кода $RM(r,m)$
$(r\leqslant 2,r<m)$. Тогда для любой невырожденной матрицы $H$
справедливо:
\begin{itemize}
\item[1)] каждое множество $\mathcal L(H,HD)$ $(D\in \mathcal A(RM(r,m)))$
содержит только один $A$-класс $A[(H,HD)]$. Причём мощность этого класса равна
$|Aut(RM(r,m)|^2$, если $H$ принадлежит множеству $A(RM(r,m)))$, и~---
$2|Aut(RM(r,m))|^2$ иначе;
\item[2)] каждое множество $\mathcal L(H,HT_{\widetilde\alpha})$ содержит
в точности
$$2^{\dim\left[RM^1_{\supp((1,\widetilde\alpha)R)}(r,m)\right]-1}$$
$A$-классов мощности $2|Aut(RM(r,m))|^2$;
\item[3)] каждое множество $\mathcal
L(H,HT^{i}_{\widetilde\alpha})(T^{i}_{\widetilde\alpha}\not\in\mathcal
A(RM(r,m)))$ содержит не менее
$$2^{\dim\left[RM^i_{I}(r,m)\right]-1}$$
$A$-классов мощности $2|Aut(RM(r,m))|^2$; здесь под $I$ понимается
множество $\overline{\supp}(\overrightarrow{R_i}\oplus
\overrightarrow{T^{i}_{\widetilde\alpha}R}_i)$, а
$\overrightarrow{R_i}$ и
$\overrightarrow{T^{i}_{\widetilde\alpha}R}_i$ --- $i$-тые строки
матриц $R$ и $T^{i}_{\widetilde\alpha}R$ соответственно.
\end{itemize}
\end{theorem}

\begin{theorem}\label{theorem2}
Пусть $R$ --- стандартная форма порождающей матрицы кода $RM(r,m)$
$(r\leqslant 2,r<m)$. Представим некоторую перестановку $\Gamma$ в
виде произведения перестановок
$\widetilde\Gamma(\Gamma_1\|\Gamma_2)$, где $\widetilde\Gamma\in
S_{2n}$, $\Gamma_1,\ldots,\Gamma_2\in S_n$. Тогда для любой
невырожденной матрицы $H$ справедливо:
\begin{itemize}
\item[1)] каждое множество $\mathcal G(H,HD)$ $(D\in \mathcal A(RM(r,m)))$
содержит перестановку  $\Gamma$, если и только если
$\Gamma_1,\Gamma_2\in Aut(RM(r,m))$, а перестановка
$\widetilde\Gamma$ выделяет блоки одинаковых столбцов и переводит их
друг в друга;
\item[2)] каждое множество $\mathcal G(H,HT_{\widetilde\alpha})$
содержит перестановку  $\Gamma$, если и только если
$\Gamma_1,\Gamma_2\in Aut(RM(r,m))$, а перестановка
$\widetilde\Gamma$ либо
\begin{itemize}
\item[i)]выделяет блоки одинаковых столбцов и переводит их друг в
друга, либо
\item[ii)]выбирает некоторый вектор $x$ из кода
$RM^1_{\supp((1,\widetilde\alpha)R)}(r,m)$ и переводит столбцы
$HR_i$ и $HT_{\widetilde\alpha}R_i$ друг в друга для любого $i\in
\supp(x)$, либо
\item[iii)] i) и ii) вместе;
\end{itemize}
\item[3)] Если перестановка $\Gamma$ такова, что $\Gamma_1,\Gamma_2\in
Aut(RM(r,m))$, а $\widetilde\Gamma$
\begin{itemize}
\item[i)]выделяет блоки одинаковых столбцов и переводит их друг в
друга, либо
\item[ii)]выбирает некоторый вектор $x$ из кода
$RM^i_{I}(r,m)$ и переводит столбцы $HR_j$ и
$HT^{i}_{\widetilde\alpha}R_j$ друг в друга для любого $j\in
\supp(x)$, здесь под $I$ понимается множество
$\overline{\supp}(\overrightarrow{R_i}\oplus
\overrightarrow{T^{i}_{\widetilde\alpha}R}_i)$, а
$\overrightarrow{R_i}$ и
$\overrightarrow{T^{i}_{\widetilde\alpha}R}_i$ --- $i$-тые строки
матрицы $R$ и $T^{i}_{\widetilde\alpha}R$ соответственно; либо
\item[iii)] i) и ii) вместе,
\end{itemize}
то множество $\mathcal G(H,HT^i_{\widetilde\alpha})$ содержит
перестановку $\Gamma$.
\end{itemize}
\end{theorem}

Итак, теоремы~\ref{theorem1} и~\ref{theorem2} дают полное описание
множеств $\mathcal G(H_1,H_2)$ для всех матриц $H_1,H_2$ таких,
что $H^{-1}_1H_2$ --- либо автоморфизм кода $RM(r,m)$, либо одна
из матриц $T_{\widetilde\alpha}$. А для $\mathcal
G(H,HT^i_{\widetilde\alpha})$ описывается некоторое его
подмножество.
