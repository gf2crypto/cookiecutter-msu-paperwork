%!TEX root = {{cookiecutter.project_name}}.tex


% Информация о типе работы
\newcommand{\Year}{%
    % 2006%
    \the\year%     % Текущий год
}

% Укажите тип работы
% Например:
%     Выпускная квалификационная работа,
%     Магистерская диссертация,
%     Курсовая работа, реферат и т.п.
\newcommand{\WorkType}{%
    % Выпускная квалификационная работа%
    % Магистерская диссертация%
    % Курсовая работа%
    % Реферат%
    Дипломная работа%
}

% Название работы
%%%%%%%%%%% ВНИМАНИЕ! %%%%%%%%%%%%%%%%
% В МГУ ОНО ДОЛЖНО В ТОЧНОСТИ
% СООТВЕТСТВОВАТЬ ВЫПИСКЕ ИЗ ПРИКАЗА
% УТОЧНИТЕ НАЗВАНИЕ В УЧЕБНОЙ ЧАСТИ
\newcommand{\Title}{%
    Ключевое пространство криптосистемы Мак-Элиса--Сидельникова%
}


% Имя автора работы
\newcommand{\Author}{%
    Чижов Иван Владимирович%
}

% Информация о научном руководителе
%% Фамилия Имя Отчество%
\newcommand{\SciAdvisor}{%
    Карпунин Григорий Анатольевич%
}
%% В формате: И.~О.~Фамилия%
\newcommand{\SciAdvisorShort}{%
    Г.~А.~Карпунин%
}
%% должность научного руководителя
\newcommand{\Position}{%
    % профессор%
    доцент%
    % старший преподаватель%
    % преподаватель%
    % ассистент%
    % ведущий научный сотрудник%
    % старший научный сотрудник%
    % научный сотрудник%
    % младший научный сотрудник%
}
%% учёная степень научного руководителя
\newcommand{\AcademicDegree}{%
    % д.ф.-м.н.%
    % д.т.н.%
    к.ф.-м.н.%
    % к.т.н.%
    % без степени%
}

% Информация об организации, в которой выполнена работа
%% Город
\newcommand{\Place}{%
    Москва%
}
%% Университет
\newcommand{\Univer}{%
    Московский государственный университет имени М.~В.~Ломоносова%
}
%% Факультет
\newcommand{\Faculty}{%
    Факультет вычислительной математики и кибернетики%
}
%% Кафедра    
\newcommand{\Department}{%
    Кафедра информационной безопасности%
}     

%%%% Переключите статус документа для отладки
%%%% В режиме draft документ собирается очень быстро
%%%% и выводится полезная информация о том
%%%% какие строки вылезают за границы документа, что удобно для борьбы с ними
\newcommand{\Status}{%
    % draft%
    final%
}
